%%% Dokumentenkopf
%%% ==============
%% Konfiguration des Dokumentes
%% - Laden von Paketen, die die Formatierung beeinflussen/das Erstellen des
%%   Dokumentes vereinfachen

\ifluatex%
    % für den Einsatz mit luaTeX:
    % -------- Eingabecodierung
    %          es gibt keine spezielle Eingabecodierung, da luaTex mit UTF-8
    %          Zeichen arbeitet
    % -------- Festlegung der Fontcodierung
    \usepackage{fontspec}       % stellt über interne Makros alle
                                % nützlichen Fonts bereit
    \defaultfontfeatures{Ligatures=TeX} % unterstützt spezielle
                                % \TeX-Eigenschaften
\else

% -------- Festlegung der Fontcodierung
    \usepackage[T1]{fontenc}    % es gibt für spezielle Anwendungen
                                % spezielle Fonts, die anders codiert sind
                                % (7 bit, ...) T1: 256 Zeichen im Font
\fi%

\usepackage[german]{babel}              % lädt explizit Sprachpakete mit den
                                % Trennmustern der jeweiligen Sprache,
                                % dient der Unterstützung einfacherer
                                % Eingabe
                               

\usepackage{enumerate}			%zur einfachen Neudefinition von Aufzählungen
\usepackage{marvosym}			%enthält viele Sonderzeichen (\PointingHand zB)
\usepackage{booktabs}			%für Linien um Tabelle (\toprule, \bottomrule, \midrule)
\usepackage{array}				%für Neudefinition von Spalten
\usepackage{tabularx}			%für Tabellen mit vorgegebener Breite
%\usepackage{siunitx}			%für Tabellen mit numerischen Daten und Einheiten
\usepackage{longtable}			%für "lange Tabellen"

\usepackage{verbatim} 			%für verbatiminput
\usepackage{moreverb}			%für verbatimwrite
\usepackage{fancyvrb}			%für \DefineShortVerb
\usepackage{listings}			%für lstinputlisting u. a.
%\usepackage{listingx} 			%für lstx etc...
\usepackage[pict2e]{struktex}			%für Struktogramme
\usepackage{graphicx}			%für \includegraphics
%\usepackage{pgf-uml}			%für UML-Diagramme
\usepackage{pgfplots}			%für axis-Umgebung
	\pgfplotsset{trig format plots=rad} %Winkelfunktionen im Bogenmaß
\usepackage{amssymb}
\usepackage{stmaryrd}
\usepackage{amsmath}
\usepackage{ifthen}
\usepackage{multicol} 			% statt der Option twocolumn
\usepackage{telprint}
%\usepackage{bookmark}			%wg Parent2-Fehler

\usepackage{svg}
\usepackage{xparse}
\usepackage{wrapfig}

% für maketile
\title{Ableitungen}
\author{Felix Bengsch}
\date{\today}

% Zähler für Listen
\newcounter{myListCounter}

% Festlegen der Zählung von Abschnitten
\setcounter{secnumdepth}{5}

% Festlegen der Zählung im Inhaltsverzeichnis
\setcounter{tocdepth}{5}

% Für Baumdiagramme
\tikzset{directory/.style={rectangle,rounded corners, draw, font=\ttfamily,fill=blue!20}}
\tikzset{file/.style={rectangle, draw, font=\ttfamily, fill=green!20}}

%Fallunterscheidung
\newif\ifisreport
\isreportfalse

%für Auszählungszeichen in der Präsentation
\makeatletter
\@ifclassloaded{beamer}{
	\NewDocumentCommand{\ibBullet}{}{\makebox[0pt][r]{\textbullet\hspace{0.5em}}}
}{
	\NewDocumentCommand{\ibBullet}{}{}
	\let\frametitle=\@gobble
	\pagestyle{empty}
}
\makeatother

\usepackage[german, onelanguage]{algorithm2e}
\SetKwInput{KwData}{Variablen}


\usepackage[utf8]{inputenc}
\usepackage{subcaption} 
\usepackage[backend=biber, 
autolang=hyphen, 
style=alphabetic,
citestyle=alphabetic, 
giveninits=false ]{biblatex}
\usepackage[babel,german=quotes]{csquotes} 
\usepackage[export]{adjustbox} 
\usepackage{textcomp}
\usepackage{pgfplotstable}

%\usepackage[a4paper,top=1.2cm,left=2cm,right=2cm,bottom=2cm]{geometry}
\usepackage{mathptmx} 
\renewcommand{\familydefault}{\rmdefault}
\usepackage[onehalfspacing]{setspace}

\usepackage{chngcntr}
\setlength{\skip\footins}{1cm}

\usepackage{varioref}			%für \vref
\usepackage{cleveref}			%für \cref, nicht mit fancyref kompatibel
%für lstlisting mit cref
\renewcommand\lstlistingname{Quellcode}
\crefname{listing}{Quellcode}{listings}
\Crefname{listing}{Quellcode}{listings}


\usepackage{enumitem}
\setlist[itemize]{noitemsep, topsep=0pt}

% Farben definieren
\definecolor{codeGray}{RGB}{240,240,240}
\definecolor{codeBlack}{RGB}{0,0,0}
\definecolor{codeRed}{RGB}{221,0,0}
\definecolor{codeBlue}{rgb}{0,0,187}
\definecolor{codeYellow}{RGB}{255,128,0}
\definecolor{codeGreen}{RGB}{0,119,0}

% … und zuweisen
\lstset{%
	inputencoding=utf8,
	extendedchars=true,
	literate=%
	{ü}{{\"u}}1
	{ä}{{\"a}}1
	{Ü}{{\"U}}1,
	language=PHP,%
	%
	% Farben, diktengleiche Schrift
	backgroundcolor={\color{codeGray}},% 
	basicstyle={\small\ttfamily\color{codeGreen}},% 
	commentstyle={\color{codeYellow}},%
	keywordstyle={\color{codeBlue}},%
	stringstyle={\color{codeRed}},%
	identifierstyle={\color{codeBlue}},%
	%
	% Zeilenumbrüche aktivieren, Leerzeichen nicht hervorheben
	breaklines=true,%
	showstringspaces=false,%
	% 
	% Listing-Caption unterhalb (bottom)
	captionpos=b,%
	% 
	% Listing einrahmen
	frame=single,%
	rulecolor={\color{codeBlack}},%
	% 
	% winzige Zeilennummern links
	numbers=left,%
	numberstyle={\tiny\color{codeBlack}},%
	tabsize=4
}

\usepackage{todonotes}
\usepackage{wrapfig}
\usepackage{pdfpages}
\usepackage{forest}
\usepackage{float}

\usepackage{amsthm}
\theoremstyle{remark}
\newtheorem{poem}{Aufzählung}
\crefname{poem}{Aufzählung}{Aufzählungen}