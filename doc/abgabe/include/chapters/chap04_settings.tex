\chapter{Einstellungen}
\label{chap:settings}

Über das Einstellugnsmenü lässt sich die App in zwei Punkten konfigurieren:

\begin{enumerate}
	\item Einstellung der Sprache (s. \cref{sec:lang})
	\item Einstellung der Eingabemethode (s. \cref{sec:input})
\end{enumerate}

Die Einstellungsseite wird je nach Ansicht, in der sich der Nutzer aktuell befindet, auf verschiedene Arten geöffnet:

\begin{enumerate}
	\item \textbf{Profilverwaltung:} Die AppBar der Profilverwaltung enthält einen Button, der aus einem Zahnrad-Icon besteht und per direktem Klick die Einstellungsansicht öffnet (s. \cref{img:openSettings} links).
	\item \textbf{Sonstige Ansichten:} Alle weiteren Ansichten (außer der Profilverwaltung) enthalten in der AppBar ein Dropdown (s. \cref{img:openSettings} Mitte), mithilfe dessen über den zweiten Menüpunkt die Einstellungsseite geöffnet werden kann (s. \cref{img:openSettings} rechts).
\end{enumerate}

\begin{figure}[H]
	\centering
	\begin{subfigure}[b]{0.3\textwidth}
		\centering
		\includegraphics[width=1\textwidth]{../include/images/settings/Open_Settings_01}
	\end{subfigure}
	\hfill
	\begin{subfigure}[b]{0.3\textwidth}
		\includegraphics[width=1\textwidth]{../include/images/settings/Open_Settings_02}
	\end{subfigure}
	\hfill
	\begin{subfigure}[b]{0.3\textwidth}
		\includegraphics[width=1\textwidth]{../include/images/settings/Open_Settings_03}
	\end{subfigure}
	\caption{Links: Profilverwaltung, Mitte: Button für Dropdown-Menü, Rechts: Dropdown-Menü}
	\label{img:openSettings}
\end{figure}

	\section{Sprache}
	\label{sec:lang}
	
	Die App ist in den Sprachen Deutsch und Englisch verfügbar. Die Einstellung wird standardmäßig auf die im Gerät eingestellte Sprache gesetzt und kann manuell verändert werden (s. \cref{img:settings} rechts)
	
	\section{Eingabemethode}
	\label{sec:input}
	
	Neue Werte, die von der App simuliert werden sollen, werden über die Ansicht ``Simulation'' eingegeben. Hierfür kann in den Einstellungen angepasst werden, ob die Werte mit Textfeldern oder Slidern eingegeben werden sollen (s. \cref{img:settings} links).
	
	\begin{figure}[H]
		\centering
		\begin{subfigure}[b]{0.45\textwidth}
			\centering
			\includegraphics[width=1\textwidth]{../include/images/settings/settings01}
		\end{subfigure}
		\hfill
		\begin{subfigure}[b]{0.45\textwidth}
			\includegraphics[width=1\textwidth]{../include/images/settings/settings02}
		\end{subfigure}
		\caption{Links: Einstellung der Eingabemethode, Rechts: Einstellung der Sprache}
		\label{img:settings}
	\end{figure}